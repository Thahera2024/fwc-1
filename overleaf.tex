\documentclass[12pt]{article}
\usepackage{graphicx}
%\documentclass[journal,12pt,twocolumn]{IEEEtran}
\usepackage[none]{hyphenat}
\usepackage{graphicx}
\usepackage{listings}
\usepackage[english]{babel}
\usepackage{graphicx}
\usepackage{caption}
\usepackage{hyperref}
\usepackage{booktabs}
\usepackage{array}
\usepackage{amsmath}   % for having text in math mode
\usepackage{listings}
\usepackage{amssymb}
\lstset{
  frame=single,
  breaklines=true
}
%New macro definitions
\newcommand{\mydet}[1]{\ensuremath{\begin{vmatrix}#1\end{vmatrix}}}
\providecommand{\brak}[1]{\ensuremath{\left(#1\right)}}

\newcommand{\solution}{\noindent \textbf{Solution: }}
\newcommand{\myvec}[1]{\ensuremath{\begin{pmatrix}#1\end{pmatrix}}}
\let\vec\mathbf

\begin{document}

\begin{center}
\textbf\large{CHAPTER-1 \\  RELATIONS AND FUNCTIONS}
\end{center}
 
\section*{EXERCISE - 7.1}
\textbf{1.4 Composition of Functions and Invertible Function}

In this section, we will study composition of functions and the inverse of an obijective function. Consider the set A of all students, who appeared in Class X of a Board Examination in 2006. Each student appearing in the Board Examination is assigned a roll number by the Board which is written by the students in the answer script at the time of examination. In order to have confidentiality, the Board arranges to deface the roll numbers of students in the answer scripts and assigns a fake code number to each roll number. Let \( B \subset \mathbb{N} \) be the set of all roll numbers and \( C \subset \mathbb{N} \) be the set of all code numbers. This gives rise to two functions f : A→ B and g : B → C given by f(a) = the roll number assigned to the student a and g(b) = the code number assigned to the roll number b. In this process each student is assigned a roll number through the function f and each roll number is assigned a code number through the function g. Thus, by the combination of these two functions, each student is eventually attached a code number.
This leads to the following definition: \newline
\textbf{Definition 8:} Let \( f: A \to B \) and \( g: B \to C \) be two functions. Then the composition of \( f \) and \( g \), denoted by \( g \circ f \), is defined as the function:
\( g \circ f \) : \( A \to C \) \text{ given by } y
\[\quad (g \circ f)(x) = g(f(x)), \quad \forall x \in A.
\]. 
\\
\textbf{Example 15:} Let f : {2, 3, 4, 5} → {3, 4, 5, 9} and g : {3, 4, 5, 9} → {7, 11, 15} be functions defined as f(2) = 3, f(3) = 4, f(4) = f(5) = 5 and g (3) = g (4) = 7 and g (5) = g (9) = 11. Find \( g \circ f \) .\newline
\textbf{Solution:} We have \( g \circ f \)(2) = g (f(2)) = g (3) = 7, \( g \circ f \) (3) = g (f(3)) = g (4) =7,
\( g \circ f \)(4) = g (f(4)) = g (5) = 11 and \( g \circ f \)(5) = g (5) = 11.\newline
\textbf{Example 16:}  
Find \( g \circ f \) and \( f \circ g \), if \( f: \mathbb{R} \to \mathbb{R} \) and \( g: \mathbb{R} \to \mathbb{R} \) are given by:f(x) = \( f(x) = \cos x \), \( g(x) = 3x^2 \).Show that \( g \circ f \neq f \circ g \).\newline
\textbf{Solution:}  
We have:
\(
(g \circ f)(x) = g(f(x)) = g(\cos x) = 3 (\cos x)^2 = 3 \cos^2 x.
\)
Similarly,
\(
(f \circ g)(x) = f(g(x)) = f(3x^2) = \cos (3x^2).
\)
Note that \( 3 \cos^2 x \neq \cos(3x^2) \), for example, when \( x = 0 \).  
Hence, 
\(
g \circ f \neq f \circ g.
\)\newline
\textbf{Example 17:}  
Show that if \( f: \mathbb{R} \setminus \left\{\frac{7}{5}\right\} \to \mathbb{R} \setminus \left\{\frac{3}{5}\right\} \) is defined by:
\(
f(x) = \frac{3x + 4}{5x - 7},
\)
and \( g: \mathbb{R} \setminus \left\{\frac{3}{5}\right\} \to \mathbb{R} \setminus \left\{\frac{7}{5}\right\} \) is defined by:
\(
g(x) = \frac{7x + 4}{5x - 3},
\)
then \( f \circ g = I_A \) and \( g \circ f = I_B \), where:
\(
A = \mathbb{R} \setminus \left\{\frac{3}{5}\right\}, \quad B = \mathbb{R} \setminus \left\{\frac{7}{5}\right\}.
\)
Here, \( I_A(x) = x, \forall x \in A \) and \( I_B(x) = x, \forall x \in B \) are the identity functions on sets \( A \) and \( B \), respectively.\newline

\textbf{Solution:}  

We have:
\[
(g \circ f)(x) = g \left( \frac{3x+4}{5x-7} \right) = \frac{7 \left( \frac{3x+4}{5x-7} \right) + 4}{5 \left( \frac{3x+4}{5x-7} \right) - 3}
\]

\(
= \frac{21x+28 + 20x - 28}{15x + 20 - 15x + 21} = \frac{41x}{41} = x.
\)

Similarly,

\(
(f \circ g)(x) = f \left( \frac{7x+4}{5x-3} \right) = \frac{3 \left( \frac{7x+4}{5x-3} \right) + 4}{5 \left( \frac{7x+4}{5x-3} \right) - 7}
\)

\(
= \frac{21x+12 + 20x - 12}{35x + 20 - 35x + 21} = \frac{41x}{41} = x.
\)

Thus, 
\(
(g \circ f)(x) = x, \quad \forall x \in B, \quad \text{and} \quad (f \circ g)(x) = x, \quad \forall x \in A,
\)
which implies that:
\(
g \circ f = I_B \quad \text{and} \quad f \circ g = I_A.
\)\newline

\textbf{Example 18:}  
Show that if \( f: A \to B \) and \( g: B \to C \) are one-one, then \( g \circ f: A \to C \) is also one-one.\newline

\textbf{Solution:}  

Suppose:
\(
(g \circ f)(x_1) = (g \circ f)(x_2).
\)

\[
\Rightarrow g(f(x_1)) = g(f(x_2)).
\]

Since \( g \) is one-one, we get:
\[
f(x_1) = f(x_2).
\]

Since \( f \) is also one-one, it follows that:
\[
x_1 = x_2.
\]

Hence, \( g \circ f \) is one-one.\newline

\textbf{Example 19:}  
Show that if \( f: A \to B \) and \( g: B \to C \) are onto, then \( g \circ f: A \to C \) is also onto.\newline

\textbf{Solution:}  

Given an arbitrary element \( z \in C \), there exists a pre-image \( y \) of \( z \) under \( g \) such that:
\[
g(y) = z, \quad \text{since } g \text{ is onto.}
\]

Further, for \( y \in B \), there exists an element \( x \in A \) such that:
\[
f(x) = y, \quad \text{since } f \text{ is onto.}
\]

Therefore, 
\[
(g \circ f)(x) = g(f(x)) = g(y) = z,
\]
showing that \( g \circ f \) is onto.\newline

\textbf{Example 20:}  
Consider functions \( f \) and \( g \) such that the composite \( g \circ f \) is defined and is one-one. Are \( f \) and \( g \) both necessarily one-one?\\

\textbf{Solution:}  

Consider \( f: \{1, 2, 3, 4\} \to \{1, 2, 3, 4, 5, 6\} \) defined as:
\[
f(x) = x, \quad \forall x.
\]

And \( g: \{1, 2, 3, 4, 5, 6\} \to \{1, 2, 3, 4, 5, 6\} \) defined as:
\[
g(x) = x, \quad \text{for } x = 1, 2, 3, 4, \quad \text{and} \quad g(5) = g(6) = 5.
\]

Then, 
\[
(g \circ f)(x) = x, \quad \forall x,
\]
which shows that \( g \circ f \) is one-one. However, \( g \) is clearly not one-one, as \( g(5) = g(6) \). 

Thus, it is not necessary for both \( f \) and \( g \) to be one-one for \( g \circ f \) to be one-one.\\

\textbf{Example 21:}  
Are \( f \) and \( g \) both necessarily onto, if \( g \circ f \) is onto?\\

\textbf{Solution:}  

Consider \( f: \{1, 2, 3, 4\} \to \{1, 2, 3, 4\} \) and  
\( g: \{1, 2, 3, 4\} \to \{1, 2, 3\} \) defined as:  

\[
f(1) = 1, \quad f(2) = 2, \quad f(3) = f(4) = 3
\]
\[
g(1) = 1, \quad g(2) = 2, \quad g(3) = g(4) = 3
\]

It can be seen that \( g \circ f \) is onto, but \( f \) is not onto.
\\
\textbf{Remark:}  
- It can be verified in general that \( g \circ f \) being one-one implies that \( f \) is one-one.
- Similarly, \( g \circ f \) being onto implies that \( g \) is onto.

---
\\

\textbf{Example 22:}  
Let \( f : \{1, 2, 3\} \to \{a, b, c\} \) be a one-one and onto function given by:

\[
f(1) = a, \quad f(2) = b, \quad f(3) = c.
\]

Show that there exists a function \( g: \{a, b, c\} \to \{1, 2, 3\} \) such that \( g \circ f = I_X \) and \( f \circ g = I_Y \), where:

\[
X = \{1, 2, 3\}, \quad Y = \{a, b, c\}.
\]
\\
\textbf{Solution:}  

Consider \( g: \{a, b, c\} \to \{1, 2, 3\} \) defined as:

\[
g(a) = 1, \quad g(b) = 2, \quad g(c) = 3.
\]

It is easy to verify that the composite \( g \circ f = I_X \) is the identity function on \( X \), and \( f \circ g = I_Y \) is the identity function on \( Y \).

---
\newline
\textbf{Definition 9:}  
A function \( f: X \to Y \) is defined to be \textbf{invertible} if there exists a function \( g: Y \to X \) such that:

\[
g \circ f = I_X \quad \text{and} \quad f \circ g = I_Y.
\]

The function \( g \) is called the \textbf{inverse} of \( f \) and is denoted by \( f^{-1} \).

Thus, if \( f \) is invertible, then \( f \) must be one-one and onto. Conversely, if \( f \) is one-one and onto, then \( f \) must be invertible.

---
\\

\textbf{Example 23:}  
Let \( f: \mathbb{N} \to Y \) be a function defined as:

\[
f(x) = 4x + 3,
\]

where:

\[
Y = \{y \in \mathbb{N} \mid y = 4x + 3 \text{ for some } x \in \mathbb{N} \}.
\]

Show that \( f \) is invertible. Find the inverse.\\

\textbf{Solution:}  

Consider an arbitrary element \( y \in Y \). By definition of \( Y \), we have:

\[
y = 4x + 3 \quad \text{for some } x \in \mathbb{N}.
\]

Solving for \( x \):

\[
x = \frac{y - 3}{4}.
\]

Define \( g: Y \to \mathbb{N} \) by:

\[
g(y) = \frac{y - 3}{4}.
\]

Now,

\[
(g \circ f)(x) = g(f(x)) = g(4x + 3) = \frac{4x + 3 - 3}{4} = x.
\]

Similarly,

\[
(f \circ g)(y) = f(g(y)) = f\left(\frac{y - 3}{4}\right) = 4 \times \frac{y - 3}{4} + 3 = y - 3 + 3 = y.
\]

This shows that \( g \circ f = I_{\mathbb{N}} \) and \( f \circ g = I_Y \), which implies that \( f \) is invertible and \( g \) is the inverse of \( f \).

---
\\

\textbf{Example 24:}  
Let \( Y = \{ n^2 \mid n \in \mathbb{N} \} \subset \mathbb{N} \). Consider \( f: \mathbb{N} \to Y \) as:

\[
f(n) = n^2.
\]

Show that \( f \) is invertible and find the inverse.
\\

\textbf{Solution:}  

An arbitrary element \( y \in Y \) is of the form \( y = n^2 \) for some \( n \in \mathbb{N} \). This implies that:

\[
n = \sqrt{y}.
\]

Define \( g: Y \to \mathbb{N} \) by:

\[
g(y) = \sqrt{y}.
\]

Now,

\[
(g \circ f)(n) = g(n^2) = \sqrt{n^2} = n,
\]

and

\[
(f \circ g)(y) = f(\sqrt{y}) = (\sqrt{y})^2 = y.
\]

Thus, \( g \circ f = I_{\mathbb{N}} \) and \( f \circ g = I_Y \), proving that \( f \) is invertible with \( f^{-1} = g \).

---
\\
\textbf{Example 25:}  
Let \( f': \mathbb{N} \to \mathbb{R} \) be a function defined as:

\[
f'(x) = 4x^2 + 12x + 15.
\]

Show that \( f: \mathbb{N} \to S \), where \( S \) is the range of \( f \), is invertible. Find the inverse.\\

\textbf{Solution:}  

Let \( y \) be an arbitrary element of the range \( f \). Then:

\[
y = 4x^2 + 12x + 15 \quad \text{for some } x \in \mathbb{N}.
\]

Rewriting,

\[
y = (2x + 3)^2 + 6.
\]

Solving for \( x \),

\[
x = \frac{\sqrt{y - 6} - 3}{2}, \quad \text{as } y \geq 6.
\]

Define \( g: S \to \mathbb{N} \) by:

\[
g(y) = \frac{\sqrt{y - 6} - 3}{2}.
\]

Now,

\[
(g \circ f)(x) = g(f(x)) = g(4x^2 + 12x + 15) = g((2x + 3)^2 + 6) = \frac{\sqrt{(2x+3)^2 + 6 - 6} - 3}{2} = x.
\]

Similarly,

\[
(f \circ g)(y) = f\left(\frac{\sqrt{y - 6} - 3}{2}\right) = y.
\]

Thus, \( g \circ f = I_{\mathbb{N}} \) and \( f \circ g = I_S \), proving that \( f \) is invertible with \( f^{-1} = g \).\\

\textbf{Example 26:}  
Consider the functions \( f: \mathbb{N} \to \mathbb{N} \), \( g: \mathbb{N} \to \mathbb{N} \), and \( h: \mathbb{N} \to \mathbb{R} \) defined as:

\[
f(x) = 2x, \quad g(y) = 3y + 4, \quad h(z) = \sin z, \quad \forall x, y, z \in \mathbb{N}.
\]

Show that \( h \circ ( g \circ f ) = (h \circ g) \circ f \).\\

\textbf{Solution:}  

We have:

\[
h \circ ( g \circ f )(x) = h( g \circ f (x)) = h(g(f(x))).
\]

Substituting \( f(x) = 2x \):

\[
h(g(2x)) = h(3(2x) + 4) = h(6x + 4) = \sin(6x + 4), \quad \forall x \in \mathbb{N}.
\]

Also,

\[
((h \circ g) \circ f) (x) = (h \circ g)(f(x)) = h(g(2x)).
\]

Since we already calculated \( h(g(2x)) = \sin(6x + 4) \), we conclude:

\[
h \circ ( g \circ f ) = (h \circ g) \circ f.
\]

Thus, the result holds in this case.

---
\\

\textbf{Theorem 1:}  
If \( f: X \to Y \), \( g: Y \to Z \), and \( h: Z \to S \) are functions, then:

\[
h \circ ( g \circ f ) = (h \circ g) \circ f.
\]\\

\textbf{Proof:}  

We have:

\[
h \circ ( g \circ f )(x) = h( g \circ f (x)) = h(g(f(x))), \quad \forall x \in X.
\]

On the other hand:

\[
(h \circ g) \circ f (x) = h \circ g (f (x)) = h(g(f(x))), \quad \forall x \in X.
\]

Since both expressions are equal, we conclude:

\[
h \circ ( g \circ f ) = (h \circ g) \circ f.
\]

This completes the proof. \(\square\)\\

\textbf{Example 27:}  
Consider the functions \( f: \{1, 2, 3\} \to \{a, b, c\} \) and \( g: \{a, b, c\} \to \{\text{apple, ball, cat}\} \) defined as:

\[
f(1) = a, \quad f(2) = b, \quad f(3) = c, \quad g(a) = \text{apple}, \quad g(b) = \text{ball}, \quad g(c) = \text{cat}.
\]

Show that \( f, g \), and \( g \circ f \) are invertible. Find \( f^{-1}, g^{-1} \), and \( (g \circ f)^{-1} \), and show that:

\[
(g \circ f)^{-1} = f^{-1} \circ g^{-1}.
\]
\\
\textbf{Solution:}  

By definition, \( f \) and \( g \) are obijective functions. Define their inverses:

\[
f^{-1}: \{a, b, c\} \to \{1, 2, 3\}, \quad g^{-1}: \{\text{apple, ball, cat}\} \to \{a, b, c\}
\]

as follows:

\[
f^{-1}(a) = 1, \quad f^{-1}(b) = 2, \quad f^{-1}(c) = 3
\]

\[
g^{-1}(\text{apple}) = a, \quad g^{-1}(\text{ball}) = b, \quad g^{-1}(\text{cat}) = c.
\]

It is easy to verify that:

\[
f^{-1} \circ f = I_{\{1,2,3\}}, \quad f \circ f^{-1} = I_{\{a,b,c\}},
\]

\[
g^{-1} \circ g = I_{\{a,b,c\}}, \quad g \circ g^{-1} = I_D, \quad \text{where } D = \{\text{apple, ball, cat}\}.
\]

Now, \( g \circ f \) is given by:

\[
g \circ f (1) = \text{apple}, \quad g \circ f (2) = \text{ball}, \quad g \circ f (3) = \text{cat}.
\]

We define the inverse:

\[
(g \circ f)^{-1}: \{\text{apple, ball, cat}\} \to \{1, 2, 3\}
\]

as:

\[
(g \circ f)^{-1} (\text{apple}) = 1, \quad (g \circ f)^{-1} (\text{ball}) = 2, \quad (g \circ f)^{-1} (\text{cat}) = 3.
\]

It is easy to see that:

\[
(g \circ f)^{-1} \circ (g \circ f) = I_{\{1,2,3\}}, \quad (g \circ f) \circ (g \circ f)^{-1} = I_D.
\]

Thus, we have verified that \( f, g, \) and \( g \circ f \) are invertible. 

Furthermore, 

\[
f^{-1} \circ g^{-1} (\text{apple}) = f^{-1} (g^{-1} (\text{apple})) = f^{-1} (a) = 1 = (g \circ f)^{-1} (\text{apple}).
\]

\[
f^{-1} \circ g^{-1} (\text{ball}) = f^{-1} (g^{-1} (\text{ball})) = f^{-1} (b) = 2 = (g \circ f)^{-1} (\text{ball}).
\]

\[
f^{-1} \circ g^{-1} (\text{cat}) = f^{-1} (g^{-1} (\text{cat})) = f^{-1} (c) = 3 = (g \circ f)^{-1} (\text{cat}).
\]

Hence, we conclude that:

\[
(g \circ f)^{-1} = f^{-1} \circ g^{-1}.
\]

---
\\
\textbf{Theorem 2:}  
Let \( f: X \to Y \) and \( g: Y \to Z \) be two invertible functions. Then \( g \circ f \) is also invertible with:

\[
(g \circ f)^{-1} = f^{-1} \circ g^{-1}.
\]
\\
\textbf{Proof:}  

To show that \( g \circ f \) is invertible with \( (g \circ f)^{-1} = f^{-1} \circ g^{-1} \), we need to prove:

\[
(f^{-1} \circ g^{-1}) \circ (g \circ f) = I_X \quad \text{and} \quad (g \circ f) \circ (f^{-1} \circ g^{-1}) = I_Z.
\]

We have:

\[
(f^{-1} \circ g^{-1}) \circ (g \circ f) = ((f^{-1} \circ g^{-1}) \circ g) \circ f, \quad \text{by Theorem 1}
\]
\[
= (f^{-1} \circ (g^{-1} \circ g)) \circ f, \quad \text{by associativity}
\]
\[
= (f^{-1} \circ I_Y) \circ f, \quad \text{by definition of } g^{-1}
\]
\[
= I_X.
\]
Similarly, it can be shown that:\[(g \circ f) \circ (f^{-1} \circ g^{-1}) = I_Z.\]
\\
\textbf{Example 28:}  
Let \( S = \{1, 2, 3\} \). Determine whether the functions \( f: S \to S \) defined as follows have inverses. Find \( f^{-1} \), if it exists.
\\
\begin{enumerate}
    \renewcommand{\labelenumi}{\alph{enumi})}
    \item \( f = \{(1,1), (2,2), (3,3)\} \)
    \item \( f = \{(1,2), (2,1), (3,1)\} \)
    \item \( f = \{(1,3), (3,2), (2,1)\} \)
\end{enumerate}


\textbf{Solution:}

\begin{enumerate}
    \renewcommand{\labelenumi}{\alph{enumi})}

    \item It is easy to see that \( f \) is one-one and onto, so \( f \) is invertible with: \(
    f^{-1} = \{(1,1), (2,2), (3,3)\} = f.
    \)

    \item Since \( f(2) = f(3) = 1 \), \( f \) is not one-one, so it is not invertible.

    \item It is easy to see that \( f \) is one-one and onto, so \( f \) is invertible with:

    \(
    f^{-1} = \{(3,1), (2,3), (1,2)\}.
    \)
\end{enumerate}



\end{document}



